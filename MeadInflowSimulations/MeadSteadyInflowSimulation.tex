\documentclass[]{article}
\usepackage{lmodern}
\usepackage{amssymb,amsmath}
\usepackage{ifxetex,ifluatex}
\usepackage{fixltx2e} % provides \textsubscript
\ifnum 0\ifxetex 1\fi\ifluatex 1\fi=0 % if pdftex
  \usepackage[T1]{fontenc}
  \usepackage[utf8]{inputenc}
\else % if luatex or xelatex
  \ifxetex
    \usepackage{mathspec}
  \else
    \usepackage{fontspec}
  \fi
  \defaultfontfeatures{Ligatures=TeX,Scale=MatchLowercase}
\fi
% use upquote if available, for straight quotes in verbatim environments
\IfFileExists{upquote.sty}{\usepackage{upquote}}{}
% use microtype if available
\IfFileExists{microtype.sty}{%
\usepackage[]{microtype}
\UseMicrotypeSet[protrusion]{basicmath} % disable protrusion for tt fonts
}{}
\PassOptionsToPackage{hyphens}{url} % url is loaded by hyperref
\usepackage[unicode=true]{hyperref}
\hypersetup{
            pdftitle={Add reservoir inflow as new criteria to give Lake Mead managers more flexibility and independence to conserve water},
            pdfauthor={David E. Rosenberg},
            pdfborder={0 0 0},
            breaklinks=true}
\urlstyle{same}  % don't use monospace font for urls
\usepackage[margin=1in]{geometry}
\usepackage{graphicx,grffile}
\makeatletter
\def\maxwidth{\ifdim\Gin@nat@width>\linewidth\linewidth\else\Gin@nat@width\fi}
\def\maxheight{\ifdim\Gin@nat@height>\textheight\textheight\else\Gin@nat@height\fi}
\makeatother
% Scale images if necessary, so that they will not overflow the page
% margins by default, and it is still possible to overwrite the defaults
% using explicit options in \includegraphics[width, height, ...]{}
\setkeys{Gin}{width=\maxwidth,height=\maxheight,keepaspectratio}
\IfFileExists{parskip.sty}{%
\usepackage{parskip}
}{% else
\setlength{\parindent}{0pt}
\setlength{\parskip}{6pt plus 2pt minus 1pt}
}
\setlength{\emergencystretch}{3em}  % prevent overfull lines
\providecommand{\tightlist}{%
  \setlength{\itemsep}{0pt}\setlength{\parskip}{0pt}}
\setcounter{secnumdepth}{0}
% Redefines (sub)paragraphs to behave more like sections
\ifx\paragraph\undefined\else
\let\oldparagraph\paragraph
\renewcommand{\paragraph}[1]{\oldparagraph{#1}\mbox{}}
\fi
\ifx\subparagraph\undefined\else
\let\oldsubparagraph\subparagraph
\renewcommand{\subparagraph}[1]{\oldsubparagraph{#1}\mbox{}}
\fi

% set default figure placement to htbp
\makeatletter
\def\fps@figure{htbp}
\makeatother


\title{Add reservoir inflow as new criteria to give Lake Mead managers more
flexibility and independence to conserve water}
\author{David E. Rosenberg}
\date{September 5, 2021}

\begin{document}
\maketitle

\subsection{Description}\label{description}

This is an R Markdown document. This work supports the piece ``Add
reservoir inflow as new criteria to give Lake Mead managers more
flexibility and independence to conserve water.'' There are simulations
of reservoir draw down for inflow scenarios of 7 to 14 maf per year
every year. These simulations assume states and contractors do not
withdraw from their conservation accounts nor convert and conservation
account balances to meet mandatory conservation targets. Another plot
shows the reservoir release needed to stablize reservoir level for
different inflow scenarios. And a final plot shows reservoir recovery
from two starting storages. Key plots include:

\begin{enumerate}
\def\labelenumi{\arabic{enumi}.}
\item
  A line graph compares the schedules of total mandatory cutbacks by
  2008 Interim Guidelines and 2019 DCP. Dashed 1:1 lines show required
  cutbacks to avoid Dead Pool and to protect 1,025 feet (6.0 maf
  storage).
\item
  A line plot shows evolution of Lake Mead Active storage (y-axis) over
  time (x-axis) for mandatory conservation with different steady inflow
  scenarios. Pink area denotes the storage volumes/elevations of
  mandatory conservation target. Red area denotes lower storage.
  Simulations assume max DCP cutback in red area below 1,025 feet. We
  hope that if reservoir level gets that low the Lower Basin States and
  Mexico will cut back release further (depends on inflow) to stabilize
  reservoir level.
\item
  The reservoir release needed to stabilize reservoir storage at
  different reservoir elvations and inflows. This release is calculated
  by resolving the reservoir mass balance equation for release = inflow
  - evaporation.
\item
  Reservoir recovery simulations from two reservoir levels for different
  reservoir inflows and additional conservation efforts beyond mandatory
  targets.
\end{enumerate}

Data from 2019 Lower Basin Drought Contingency Plan (DCP), U.S.-Mexico
Minute 323, and CRSS.

\subsection{Requested Citation}\label{requested-citation}

David E. Rosenberg (2021), ``Add reservoir inflow as new criteria to
give Lake Mead managers more flexibility and independence to conserve
water.'' Utah State University. Logan, Utah.
\url{https://github.com/dzeke/ColoradoRiverFutures/tree/master/MeadInflowSimulation}.

\begin{verbatim}
## [1] 2
\end{verbatim}

\begin{verbatim}
## [1] 5
\end{verbatim}

\subsection{Figure 1. Mandatory Conservation Schedules by DCP and
Interim Shortage
Guidelines}\label{figure-1.-mandatory-conservation-schedules-by-dcp-and-interim-shortage-guidelines}

\includegraphics{MeadSteadyInflowSimulation_files/figure-latex/Fig1AvailWater-1.pdf}

\subsection{Figure 2. Simulation of Lake Mead active storage over time
for different scenarios of steady reservoir inflow (blue contours and
white boxes, million acre-feet per
year).}\label{figure-2.-simulation-of-lake-mead-active-storage-over-time-for-different-scenarios-of-steady-reservoir-inflow-blue-contours-and-white-boxes-million-acre-feet-per-year.}

\includegraphics{MeadSteadyInflowSimulation_files/figure-latex/Fig2MeadSimulation-1.pdf}

\subsection{Figure 3. Lake Mead releases to stabilize reservoir level
for different
inflows.}\label{figure-3.-lake-mead-releases-to-stabilize-reservoir-level-for-different-inflows.}

\includegraphics{MeadSteadyInflowSimulation_files/figure-latex/Fig3MeadSimulation-1.pdf}

\subsection{Figure 4. Lake Mead recovery from two shortage levels.
Numeric line labels (maf per year) indicate the sum of reservoir inflow
and additional conservation beyond mandatory
targets.}\label{figure-4.-lake-mead-recovery-from-two-shortage-levels.-numeric-line-labels-maf-per-year-indicate-the-sum-of-reservoir-inflow-and-additional-conservation-beyond-mandatory-targets.}

\includegraphics{MeadSteadyInflowSimulation_files/figure-latex/Fig4MeadSimulation-1.pdf}

\end{document}
